\documentclass[11pt,letterpaper]{article}
\usepackage{fullpage}
\usepackage[top=2cm, bottom=4.5cm, left=2.5cm, right=2.5cm]{geometry}
\usepackage{amsmath,amsthm,amsfonts,amssymb,amscd}
\usepackage{lastpage}
\usepackage{enumerate}
\usepackage{enumitem}
\usepackage{fancyhdr}
\usepackage{graphicx}
\usepackage{listings}
\usepackage{hyperref}
\usepackage{booktabs}
\usepackage{caption,cleveref,colortbl,csquotes,datatool,helvet,mathpazo,multirow,listings,pgfplots,xcolor}

\hypersetup{%
  colorlinks=true,
  linkcolor=blue,
  linkbordercolor={0 0 1}
}

\setlength{\parindent}{0.0in}
\setlength{\parskip}{0.05in}

% edit these
\newcommand\course{AST221H}
\newcommand\Title{Assignment 4}
\newcommand\Name{Jeff Shen} 
\newcommand\Id{1004911526} 
\newcommand\Date{Nov. 15, 2019}

\pagestyle{fancyplain}
\headheight 35pt
\lhead{\Name}
\lhead{\Name\\\Id}
\chead{\LARGE \Title}
\rhead{\course \\ \Date}
\lfoot{}
\cfoot{}
\rfoot{\small\thepage}
\pgfplotsset{compat=1.16}
\headsep 1.5em

\begin{document}

% problem 1
\section*{Problem 1: Electron degeneracy pressure}

\begin{itemize}
    \item Equation 16.12: $P = \frac{(3\pi^2)^{2/3}}{5}\frac{\hbar^2}{m_e}\left[\left(\frac{Z}{A}\right)\frac{\rho}{m_H}\right]^{5/3}$
    \item $\hbar = 1.055\times 10^{-34}~{\rm kg\,m^2\,s^{-1}}$
    \item $m_e = 9.11\times 10^{-31}~{\rm kg}$
    \item $m_H = 1.674\times 10^{-27}~{\rm kg}$
    \item $M_\odot = 1.99\times 10^{30}~{\rm kg}$
    \item $R_\odot = 6.96\times 10^{8}~{\rm m}$
\end{itemize}

\begin{enumerate}[label=(\alph*)]

    \item We can calculate the core density using the given equation:
        \begin{equation*}
            \rho_c = \frac{3M_c}{4\pi R_c^3} = \frac{3\times (0.1 \times 1.99\times 10^{30}~{\rm kg})}{4\pi \times (0.1\times 6.96\times 10^{8}~{\rm m})^3} = 1.41\times 10^{5}~{\rm kg\,m^{-3}}.
        \end{equation*}

        For hydrogen, $Z=1$, $A=1$, and for helium, $Z=2$, $A=4$, so assuming a 50-50 composition of hydrogen and helium, $\frac{Z}{A}= \frac{1+2}{1+4} = \frac{3}{5}$. Using this and the other information we have, we can calculate the electron degeneracy pressure using equation 16.12: 

            \begin{align*}
                P &= \frac{(3\pi^2)^{2/3}}{5}\frac{\hbar^2}{m_e}\left[\left(\frac{Z}{A}\right)\frac{\rho_c}{m_H}\right]^{5/3} \\a
                  &= \frac{(3\pi^2)^{2/3}}{5}\frac{(1.055\times 10^{-34}~{\rm kg\,m^2\,s^{-1}})^2}{9.11\times 10^{-31}~{\rm kg}}\left[\left(\frac{3}{5}\right)\frac{1.41\times 10^{5}~{\rm kg\,m^{-3}}}{1.674\times 10^{-27}~{\rm kg}}\right]^{5/3} \\
                  &= 1.62\times 10^{15}~{\rm N\,m^{-2}}.
            \end{align*}

        The central pressure given above can be calculated as follows:

        \begin{align*}
            P_c &= \frac{3GM_c^2}{8\pi R_c^4} \\
                &= \frac{3\times 6.67\times 10^{-11}~{\rm N\,m^2\,kg^{-2}} \times (0.1\times 1.99\times 10^{30}~{\rm kg})^2}{8\pi\times (0.1\times 6.96\times 10^{8}~{\rm m})^4} \\
                &= 1.34\times 10^{16}~{\rm N\,m^{-2}}.
        \end{align*}

        Electron degeneracy pressure accounts for approximately 10\% of the pressure in the core. This is not an insignificant amount. 

    \item Assume that the core mass remains the same. Then for core density, we have
        \begin{equation*}
            \rho'_c = \frac{3M_c}{4\pi R'_c^3} 
                 = \frac{3M_c}{4\pi\times (0.1\times R_c)^3} 
                 = \frac{1}{0.1^3}\frac{3M_c}{4\pi R_c^3} 
                 = 1000\times \rho_c 
                 = 1.41\times 10^{8}~{\rm kg\,m^{-3}},
        \end{equation*}

        and for central pressure, we have 
        \begin{equation*}
            P'_c = \frac{3GM_c^2}{8\pi R'_c^4} 
                 = \frac{3GM_c^2}{8\pi\times (0.1\times R_c)^4} 
                 = \frac{1}{0.1^4}\frac{3GM_c^2}{8\pi R_c^4} 
                 = 10000\times P_c
                 = 1.34\times 10^{20}~{\rm N\,m^{-2}}.
        \end{equation*}

    \item The degeneracy pressure, using $\rho = \rho'_c$ and $\frac{Z}{A} = \frac{2}{4} = 0.5$ for a pure helium core, is 
        \begin{equation*}
            P = \frac{(3\pi^2)^{2/3}}{5}\frac{(1.055\times 10^{-34}~{\rm kg\,m^2\,s^{-1}})^2}{9.11\times 10^{-31}~{\rm kg}}\left[\left(0.5\right)\frac{1.41\times 10^{8}~{\rm kg\,m^{-3}}}{1.674\times 10^{-27}~{\rm kg}}\right]^{5/3} = 1.19\times 10^{20}~{\rm N\,m^{-2}}.
        \end{equation*}

        At this point, electron degeneracy accounts for most of the central pressure in the star. Yes, it is important.  

    \item Again using the same formulas, we find that core density is 
        \begin{equation*}
            \rho_c = \frac{3M_c}{4\pi R_c^3} = \frac{3\times (0.1 \times 8\times 1.99\times 10^{30}~{\rm kg})}{4\pi \times (0.1\times 8\times 6.96\times 10^{8}~{\rm m})^3} = 2.2\times 10^{3}~{\rm kg\,m^{-3}}, 
        \end{equation*}

        central pressure is 
        \begin{equation*}
            P_c = \frac{3GM_c^2}{8\pi R_c^4} 
                = \frac{3\times 6.67\times 10^{-11}~{\rm N\,m^2\,kg^{-2}} \times (0.1\times 8\times 1.99\times 10^{30}~{\rm kg})^2}{8\pi\times (0.1\times 8\times 6.96\times 10^{8}~{\rm m})^4} 
                = 2.1\times 10^{14}~{\rm N\,m^{-2}},
        \end{equation*}

        and electron degeneracy pressure is 
        \begin{equation*}
            P = \frac{(3\pi^2)^{2/3}}{5}\frac{(1.055\times 10^{-34}~{\rm kg\,m^2\,s^{-1}})^2}{9.11\times 10^{-31}~{\rm kg}}\left[\left(0.5\right)\frac{2.2\times 10^{3}~{\rm kg\,m^{-3}}}{1.674\times 10^{-27}~{\rm kg}}\right]^{5/3} = 1.16\times 10^{12}~{\rm N\,m^{-2}}.   
        \end{equation*}

        Degeneracy pressure is not as important in this higher mass star as it was for the lower mass star. For a higher-mass star, because the degeneracy pressure is not as significant, gravity will be relatively stronger. More gravitational pressure means that the star will contract, leading to the higher temperatures necessary for the fusion of heavier elements. In a low mass star, degeneracy pressure will be sufficiently strong to stop the inward pull of gravity, meaning that heavier elements will not be able to fuse. 

\end{enumerate}

% problem 2
\section*{Problem 2: Hot Jupiters and tidal disruption}

\begin{enumerate}[label=(\alph*)]
    \item The point at which an object is tidally disrupted is the Roche limit. We can appproximate this by assuming (incorrectly, according to C\&O) that this is the point when the differential force is greater than the self-gravity of the planet: 
        \begin{align*}
            \frac{GM_J}{R_J^2} < \frac{2GM_SR_J}{r^3}.
        \end{align*}

        If we assume that both the Sun-like star and the Jupiter-like planet are spherical, then 
        \begin{align*}
            \rho &= \frac{M}{V} = \frac{M}{\frac{4}{3}\pi R^3} = \frac{3M}{4\pi R^3}.        \end{align*}

        Substituting this into the previous equation, cancelling terms, and rearranging, we find the following expression:
        \begin{alignat*}{2}
            &&G \left(\frac{4\pi\rho_J R_J}{3}\right) &<  \frac{2GR_J}{r^3}\left(\frac{4\pi\rho_S R_S^3}{3}\right) \\
            \implies&& \rho_J &< \frac{2\rho_S R_S^3}{r^3} \\
            \implies&& r &< \left(\frac{2\rho_S R_S^3}{\rho_J}\right)^{1/3} \\
            \implies&& r &< \left(2\,\frac{\rho_S}{\rho_J}\right)^{1/3} R_S.
        \end{alignat*}

    \item First we need to calculate the mean densities of the star and the planet. If we assume constant density everywhere, then 
        \begin{alignat*}{2}
            \rho_S &= \frac{3M_S}{4\pi R_S^3} &\qquad\qquad\qquad \rho_J &= \frac{3M_J}{4\pi R_J^3} \\
                   &= \frac{3\times 1.9891\times 10^{33}~{\rm g}}{4\pi\times (6.955\times 10^{10}~{\rm cm})^3} &&= \frac{3\times 1.90\times 10^{30}~{\rm g}}{4\pi\times (7.1492\times 10^{9}~{\rm cm})^3} \\
                   &= 1.411~{\rm g\,cm^{-3}} &&= 1.241~{\rm g\,cm^{-3}} 
        \end{alignat*}

        Then the Roche limit $r$ is given by 
        \begin{equation*}
            r = 2.456\left(\frac{1.411~{\rm g\,cm^{-3}}}{1.241~{\rm g\,cm^{-3}}}\right)^{1/3} R_S = 2.56\,R_S
        \end{equation*}

        Since this is greater than the radius of the star, this means that the tidal disruption limit is outside the host star.

        We can also write the distance in AU:
        \begin{equation*}
            r = 2.56\,R_S = 2.56\times 6.955\times 10^{10}~{\rm cm} = 1.78\times 10^{11}~{\rm cm} = 0.012~{\rm AU}.
        \end{equation*}

    \item Since we know the tidal disruption radius, and the star is a solar mass star, we can use the simplified form (probably why we did the AU calculation?) of Kepler's Third Law:
        \begin{equation*}
            P^2 = a^3,
        \end{equation*}
        where $P$ is in years and $a$ is in AU. Then 
        \begin{equation*}
            P= a^{3/2} = 0.012^{3/2} = 0.0013~{\rm years} = 11.4~{\rm hours}.
        \end{equation*}

        Since we have the masses of the star and the planet, we can use the general form of Kepler's Third Law (to verify?),
        \begin{equation*}
            P^2 = \frac{4\pi^2 a^3}{G(m_1+m_2)},
        \end{equation*}
        where $P$ is in seconds, $m_1, m_2$ are in kilograms, and $a$ is in meters. Then
        \begin{align*}
            P^2 &= \frac{4\pi^2 r^3}{G(M_S+M_J)} \\
                            &= \frac{4\pi^2\times (1.78\times 10^{9}~{\rm m})^3}{6.67\times 10^{-11}~{\rm N\,m^2\,kg^{-2}} \times (1.9891\times 10^{30}~{\rm kg} + 1.90\times 10^{27}~{\rm kg})} \\
                              &= 1.68\times 10^{9}~{\rm s}^2.
        \end{align*}

        Taking the square root and converting to hours, we find that the orbital period is
        \begin{equation*}
            P = \sqrt{1.68\times 10^{9}~{\rm s}^2} = 4.095\times 10^{4}~{\rm s} = 11.4~{\rm hours},
        \end{equation*}
        as expected.

    \item From Assignment 1, given a binary system with the assumption of circular orbits, we have a relation between the masses of the objects and their distances to the center of mass:
        \begin{equation*}
            \frac{M_S}{M_J} = \frac{a_J}{a_S}
        \end{equation*}
        where $a_J, a_S$ are the distances from the center of mass to the hot Jupiter and to the star respectively. Using this and the other given information, can express $a_S$ in terms of $a_J$: 
        \begin{equation*}
            a_S = \frac{M_J}{M_S}\times a_J = \frac{1.90\times 10^{30}~{\rm g}}{1.9891\times 10^{33}~{\rm g}}\,a_J = 0.001\,a_J \simeq 4.293\times 10^{6}~{\rm m},
        \end{equation*}
        and then find the total distance between the hot Jupiter and the star: 
        \begin{equation*}
            a = a_S + a_J = (1+0.001)\,a_J = (1.001)\times 0.03~{\rm au} = 0.03003~{\rm au} = 4.5\times 10^{9}~{\rm m}.
        \end{equation*}

        We can use Kepler's Third Law to find the orbital period: 
        \begin{align*}
            P_S^2 &= \frac{4\pi^2 a^3}{G(M_J + M_S)} \\
                  &= \frac{4\pi^2\times (4.5\times 10^{9}~{\rm m})^3}{(6.67\times 10^{-11}~{\rm N\,m^2\,kg^{-2}})\times (1.90\times 10^{27}~{\rm kg} + 1.9891\times 10^{27}~{\rm kg})} \\
                  &= 1.387\times 10^{13}~{\rm s^2}.
        \end{align*}
        Taking the square root of both sides, we find that the period of the star is
        \begin{equation*}
            P_S = \sqrt{1.387\times 10^{13}~{\rm s^2}} = 3724013~{\rm s} = 43.1~{\rm days}.
        \end{equation*}

        From this, we can calculate the orbital velocity of the star:
        \begin{equation*}
            v_S = \frac{2\pi a_S}{P_S} = \frac{2\pi\times 4.293\times 10^{6}~{\rm m}}{3724013~{\rm s}} = 7.24~{\rm ms^{-1}}.
        \end{equation*}

        Since the orbit is circular, we know that regardless of whether the observer is located, there will be two points in the orbit when the total velocity is equal to the radial velocity—once when it is moving directly towards the observer, and once when it is moving directly away from the observer. Mathematically, to find the radial velocity of the star, we only want one component of orbital velocity. $\sin{\theta}$ (or $\cos{\theta}$, depending on how you look at it) gives the relation between the radial velocity (towards the observer) and the orbital (2D) velocity:
        \begin{align*}
            \sin{\theta} = \frac{v_{Sr}}{v_S} \\
            \implies v_{Sr} = v_S\sin{\theta}.
        \end{align*}
        Since we only want the amplitude, we can find the maximum (or minimum) of $v_{Sr}$. Since $v_S$ is constant, the maxima of $v_{Sr}$ coincide with the maxima of $\sin{\theta}$, at which points it equals 1. Then $\max{(v_{Sr})} = v_S = 7.24~{\rm m\,s^{-1}}$ is the amplitude of the radial velocity of the star. 


\end{enumerate}




% problem 3
\section*{Problem 3:}
\begin{itemize}
    \item radius of Jupiter: $R_J = 69,911~{\rm km} = 6.991\times 10^{9}~{\rm cm}$
    \item mass of Jupiter: $M_J = 1.898\times 10^{30}~{\rm g}$
    \item mean distance from Sun to Jupiter: $d = 778.5 \times 10^{6}~{\rm km} = 7.785 \times 10^{13}~{\rm cm}$
    \item luminosity of Sun: $L_\odot = 3.84 \times 10^{33}~{\rm erg\,s^{-1}}$
\end{itemize}

\begin{enumerate}[label=(\alph*)]
    
    \item We can find the total energy emitted in the infrared by multiplying the flux of Jupiter by its surface area:
        \begin{equation*}
            L = F*4\pi R^2 = 1.41\times 10^{4}~{\rm erg\,s^{-1}}\times 4\pi\times (6.991 \times 10^{9}~{\rm cm})^2 = 8.66\times 10^{24}~{\rm erg\,s^{-1}}.
        \end{equation*}
   
    \item Since the distance from Jupiter to the Sun is very large relative to the radius of Jupiter, we can approximate the shape of Jupiter on which light is incident as a circle, which has area $\pi R_J^2.$ Then the rate of energy absorbed by Jupiter from solar radiation is 
    \begin{align*}
        L &= (1-A)\times \frac{L_\odot}{4\pi d^2}\times \pi R_J^2 \\
          &= (1-A)\times \frac{L_\odot R_J^2}{4d^2} \\
          &= (1-0.343)\frac{(3.84 \times 10^{33}~{\rm erg\,s^{-1}}) (6.991 \times 10^{9}~{\rm cm})^2}{4\times (7.785 \times 10^{13}~{\rm cm})^2} \\
          &= 5.09\times 10^{24}~{\rm erg\,s^{-1}}.
    \end{align*}
    
    \item The excess rate of energy emitted by Jupiter is the difference between the rate of energy emission and the rate of energy absorption: 
        \begin{align*}
            L_{excess} = 8.66\times 10^{24}~{\rm erg\,s^{-1}} - 5.09\times 10^{24}~{\rm erg\,s^{-1}} = 3.57\times 10^{24}~{\rm erg\,s^{-1}}.
        \end{align*}

    \item Assuming the mass fraction of radioactive elements is the same for Earth and for Jupiter, we get        
        \begin{equation*}
            \frac{E_{rad, \oplus}}{M_\oplus} = \frac{E_{rad, J}}{M_J}.
        \end{equation*}
        Rearranging for $E_{rad, J}$, we find that 
        \begin{equation*}
            E_{rad, J} = \frac{M_J}{M_\oplus}E_{rad, \oplus}
                       = \frac{1.898\times 10^{27}~{\rm kg}}{5.972\times 10^{24}~{\rm kg}}\times 2.2\times 10^{20}~{\rm erg\,s^{-1}} 
                       = 6.991 \times 10^{22}~{\rm erg\,s^{-1}}.
        \end{equation*}


    \item The work done to move a cloud of mass $M_J$ from $r=\infty$ to $r=R_J$ is the force ($\frac{GMM_J}{r^2}$) integrated over the displacement:
        \begin{align*}
            W = \int_{\infty}^{R_J} \frac{GM_JM_J}{r^2}dr = -\frac{GM_J^2}{r} \Big |_{\infty}^{R_J}
        \end{align*}

        Since $\displaystyle\lim_{r\to\infty}\frac{1}{r} = 0$, we can rewrite this as
        \begin{equation*}
            W = -\frac{GM_J^2}{R_J} \\
              = -\frac{(6.67\times 10^{-11}~{\rm N\,m^2\,kg^{-2}})\times (1.898\times 10^{27}~{\rm kg})^2}{6.991 \times 10^{7}~{\rm m}} \\
              = -3.44\times 10^{36}~{\rm Nm \text{ (equivalent to J)}}.
        \end{equation*}

    \item The Virial Theorem relates gravitational potential energy to kinetic energy:
        \begin{equation*}
            KE = -\frac{1}{2}U.
        \end{equation*}
        We can also relate $KE$ to $T$ using the following equation: 
        \begin{equation*}
            KE = \frac{3}{2}Nk_B T.
        \end{equation*}
        But note that the number of particles can be rewritten as the total mass divided by the mean molecular mass. Then putting these two equations together, we find that
        \begin{alignat*}{2}
            &KE = \frac{3}{2}\frac{M}{\overline{m}}k_B T = -\frac{1}{2}U \\
            \implies &T = -\frac{U\overline{m}}{3Mk_B} \\
                     & = -\frac{(-3.44\times 10^{36}~{\rm J})\times (2.2\times 1.67\times 10^{-27}~{\rm kg})}{3\times (1.898\times 10^{27}~{\rm kg})\times (1.38\times 10^{-23}~{\rm J\,K^{-1}})} \\
                     &= 160,843~{\rm K}.
        \end{alignat*}




\end{enumerate}

\end{document}
