\documentclass[twocolumn]{aastex63}
\usepackage{natbib}
\usepackage{amsmath,amsthm,amssymb}
\usepackage{floatrow}
\usepackage{setspace}
\usepackage{graphicx}
\usepackage{tabularx}
\usepackage[inline]{enumitem}
\usepackage{setspace}
\usepackage{lipsum}
\graphicspath{{./figures/}}
\bibliographystyle{aasjournal}

%\onehalfspacing
\setstretch{1.13}

\begin{document}

\author{Jeff Shen}
\title{The Formation of Planetary Nebulae: \\
Mass Loss Mechanisms of Late-Stage AGB Stars} 

%\begin{abstract}
%    \lipsum[2]\citep{testing}
%\end{abstract}

%%%%%%%%%%%%%%%%%%%%%%%% Introduction
\section{Introduction} \label{sec:intro}

The asymptotic giant branch is a region on the HR-Diagram populated by low- to intermediate-mass stars late in their lives. Stars in this region range from $0.6-10~M_\odot$. However, knowing that stars in this mass range eventually become white dwarfs, and that white dwarfs have a maximum (stable) mass of $1.4 M_\odot$ (Chandrasekhar limit), it is apparent that these stars must have lost the majority of their mass—up to several solar masses at some point in their evolution \cite{willson}. A natural question to ask would be how that star sheds its mass. 

Mass loss is of critical importance in the discussion of the formation of planetary nebulae. MS stars lose most of their mass as they evolve, and the majority of that mass loss takes place on the AGB track. This lost mass is partially composed of gases which are eventually responsible for the planetary nebulae that we see—a planetary nebula (abbreviated PN or plural PNe) is an interstellar cloud composed of ionized gases ejected from a low- to intermediate-mass star near the end of its stellar lifetime. However, not all gases appear as PNe—the vibrant colours associated with PNe are caused by ionizing ultraviolet radiation from the central star (CSPN). For the glow to be visible, the CSPN must have a temperature of at least $30,000~{\rm K}$, and the density in the cloud should be be upwards of 100 particles per ${\rm cm^{-3}}$ \citep{prialnik}.

This paper aims to explore the transition of AGB stars as they shed mass to form planetary nebulae, as well as the mechanisms that drive the transition. Despite the fact that the physics behind some of these mechanisms is not well understood quantitatively \citep{blocker}, an attempt will be made to present theories on the causes of formation of PNe.

%%%%%%%%%%%%%%%%%%%%%%%%%%%%%%% Stellar Winds
\section{Stellar Winds} \label{sec:winds}

Stellar winds are outflows of gas which cause the ejection of material in the stellar atmosphere into what is called a \textit{circumstellar envelope (CSE)}. In order for these gases to be ejected from the star such that they are no longer gravitationally bound to the (core of the) star, they must be accelerated past the escape velocity of the star. Fortunately for late-stage AGB stars, ``typical stellar radii [are] of several hundred solar radii. In combination with$\ldots$ masses well below 8 $M_\odot$, the resulting surface gravities are typically 4–5 orders of magnitude below that of a Sun-like star" \citep{hofner}. Thus, the low surface gravity means that mass loss is significantly higher in late-stage AGB stars than for Sun-like stars, or other stars earlier in their evolutionary lifetimes. The dynamics of stellar winds are complex and varied—there are coronal winds, sound-driven winds, dust-driven winds, magnetic rotating winds, Alfv\'en wave driven winds, line driven winds, etc. \citep{lamers}

For cooler stars with temperatures in the range of $\sim 3000\,K$, stellar winds are primarily of the dust-driven variation. Radiation pressure from the star imparts its momentum onto grains of dust which are able to condense in the upper atmospheres of these stars due to the lower temperatures. When dust condenses, the opacity to radiation is increased, and thus, the radiation pressure, directed radially outwards, is able to eject this dust away from the star. In this process, the dust is able to, by friction or by other methods, carry some of the surrounding gases with it away from the star. ``Small grains are adequetely coupled to the gas in the sense that they efficiently communicate the momentum they absorb from the radiation field to the gas, and that momentum is effectively diffused throughout the gas" \citep{gilman}. These dust-driven winds have terminal velocities in the range of $10-30~{\rm km\,s^{-1}}$ \citep{lamers}. Because these dust grains can absorb radiation over a broad range of wavelengths, this mechanism of mass loss is also called `continuum driven’ winds. 


Unfortunately, as reliable (theoretical) models are not available, many turn to the following empirical formula by \cite{reimers} for estimating mass loss due to stellar winds:
\begin{equation*}
    \dot M = -4\times 10^{-13}\,\eta \left(\frac{L_*R_*}{M_*}\right)~{\rm M_\odot\,yr^{-1}},~{\rm \eta \sim 1}
\end{equation*}
where $L_*, R_*$, and $M_*$ are given in solar units. This model assumes that mass loss is proportional to luminosity (radiation pressure), and inversely proportional to surface gravity. Knowing the approximate stellar parameters of late-stage AGB stars allows us to estimate the amount of mass lost: AGB stars are typically several thousand times as luminous as the Sun, several hundred times larger in radius, and have a mass of several solar masses. This roughly gives us a mass loss rate on the order of $10^{-6}~{\rm M_\odot\,yr^{-1}}$.

Many other mass loss equations, such as the following one given by \cite{schroeder} as an improvement—although it is ``not applicable to molecule-driven, dust-driven, and pulsational winds"—for estimating mass loss in giants with low gravity, are merely variations of Reimers' equation:
\begin{equation*}
    \dot M = \eta\,\frac{L_*R_*}{M_*}\,\left(\frac{T_{eff}}{4000\,K}\right)^{3.5}\left(1+\frac{g_\odot}{4300g_*}\right),
\end{equation*}
where $L_*, R_*$, and $M_*$ are defined as before, $g_*$ and $g_\odot$ are the stellar and solar surface gravity respectively, and $\eta = 8 (\pm 1) \times 10^{-14}~{\rm M_\odot\,yr^{-1}}$. 

\subsection{Superwinds} \label{subsec:superwinds}

The simple conclusion that PNe are made of the diffuse material from the CSE of an AGB star is a natural one. However, \cite{kwok2000} identifies several problems: 
\begin{enumerate*}[label=(\roman*)]
    \item the densities of PNe shells are higher than that of those in the CSE of AGB stars,
    \item the shell- and bubble-like structures of PNe are well-defined with sharp boundaries, rather than diffuse like the CSE of AGB stars, 
    \item and the observed expansion velocities of PNe are higher than that of stellar winds in AGB stars. 
\end{enumerate*}

This evidence points to a so-called ``superwind", which is a rapid ejection of material caused by a sharp increase in the rate of mass loss at the very end of an AGB star's lifetime. \cite{kwok1978} propose that as this superwind is ejected from the star at extreme velocities (on the order of $10^{3}~{\rm km\.s^{-1}}$), it crashes into the existing, previously ejected stellar material (which is travelling out at much lower speeds). The compression of the slower winds by the fast wind causes the formation of the definite structure that is seen in PNe \footnote{This is perhaps analogous to the ``snow-plow" phase of supernovae (e.g. \cite{moriya, mccray}).}

After a superwind ejects the bulk of the circumstellar envelope from the star, the conditions for the formation of a planetary nebula are nearly met. As the CSPN continues burning its fuel and contracting, it heats up. This continues on until it reaches a temperature of approximately 30,000 $K$, at which point the photons that it emits are sufficiently energetic to ionize the surrounding gases. The electrons in the atoms of the gases are excited, and as they drop down to lower energy levels, they re-radiate in the visible spectrum, giving the characteristic glow of PNe. The cause of this superwind is not entirely clear, but a potential source of the exponential increase in mass loss is the pulsation of AGB stars. 


%%%%%%%%%%%%%%%%%%%%%%%%%%%%%%%%%% Pulsation

\section{Pulsation Theory} \label{sec:pulsation}
% enrichment %
PNe play an important role in the enrichment of the interstellar medium (ISM) and galaxies. During the later stages of and AGB star's lifetime (when it is classified as a thermally-pulsating AGB star, or TP-AGB star), thermal pulses caused by the unstable double-shell burning cause metals from the core to be mixed into the outer layers of the star in a processed called \textit{dredge-up}. When a PN is formed, stellar winds carry these heavier elements—which are now closer to the surface of the star and thus are easier to expel—into the ISM \citep{iben}. 

As it turns out, this pulsation also has other consequences on the mass loss of the star. Pulsation causes fluctuations in stellar radius as the star tries to maintain equilibrium despite thermal instability. Consequently, shock waves are produced, which levitate material in the atmosphere. If we think of mass loss as the rate at which some amount of material in a spherically symmetric shell \footnote{Spherical symmetry turns out to not be a good assumption. Further discussed in \S \ref{sec:asymmetry}.} is moving away from the star, then we have the relation
\begin{equation*}
    \dot M = 4\pi r^2 \rho v.
\end{equation*}

Because pulsation causes more material to be present in the stellar atmosphere, density (and opacity) increases, and this means that there is a greater potential for mass to be lost to radiation pressure \citep{liljegren}. 

\subsection{Mira Variables} \label{subsec:mira}

Of particular interest in this discussion are Mira variables, which are a particularly promising (in terms of candidacy for being the progenitors for PNe) class of stars at the top of the AGB track with pulsation periods on the order of a year. As Miras evolve, the period of their pulsation of period increases \citep{fadeyev}. This is significant for multiple reasons. 

Mira variables are known to have a period-luminosity relation; using period and luminosity data from Mira variables in the Large Magellanic Cloud (LMC) as seen below,

\begin{figure}[ht]
    \includegraphics[width=\textwidth]{period_luminosity.png}
\end{figure}

\cite{glass} give a period-luminosity relation for Mira variables: 
\begin{equation*}
    m_{bol} = 19.25 - 2.09\log P.
\end{equation*}

This period-luminosity relation implies that Miras near the end of their lifetimes have higher luminosities. \footnote{Luminosity is written such that more negative numbers indicate greater luminosity.} This increased luminosity, being linked to increased radiation pressure, implies that mass loss by stellar winds is greater (see \S \ref{sec:winds}). 

Moreover, when a star begins pulsating with a period of 60 days, mass loss due to pulsation is initiated. After some time, as the pulsation period increases to $\sim 300$ days, the star transitions to the fundamental pulsation mode, and mass loss increases by a factor of $\sim 100$ \citep{mcdonald, bedijn}. This sharp increase in mass loss rate at $P\sim 300$ days is supported by empirical data from \cite{vassiliadis}, who observed the following between pulsation period and mass loss for Miras:
\begin{figure}[ht]
    \includegraphics[width=\textwidth]{pulsation_massloss.png}
\end{figure}

\cite{vassiliadis} give the relation 
\begin{align*}
    \log \dot M~{[\rm M_\odot\,yr^{-1}]} = -11.4 + 0.0123\,P~{[\rm days]}
\end{align*}

for stars with $M\leq 2.5M\odot$, and
\begin{align*}
    \log \dot M~{[\rm M_\odot\,yr^{-1}]} &= -11.4 \\
                                         &+ 0.0125\,\left(P~{[\rm days]} - 100\left(\frac{M_*}{M_\odot} - 2.5\right)\right)
\end{align*}

for stars with $M\geq 2.5M\odot$. From these equations, it can be seen that there is an exponential increase in the mass loss rate towards the end of an AGB star's life. This increase by a factor of $\sim 100$ from before (where we approximated the mass loss rate to be $10^{-6}~{\rm M_\odot\,yr^{-1}}$) puts the rate of mass loss at  $10^{-4}~{\rm M_\odot\,yr^{-1}}$, which happens to be approximately the mass loss rate due to superwinds \citep{iben}. 


%%%%%%%%%%%%%%%%%%%%%%%%%%%%%%%%%%%%%%%%%%% Asymmetry
\section{Asymmetry}\label{sec:asymmetry}

The complexity of interactions between all of these physical processes means that much remains unclear. There is a lack of powerful, universal theoretical models that can be used to explain how PNe form. In particular, it is unclear what gives rise to the incredible variety of shapes of PNe. Assumptions of stellar winds (and thereby PNe) being spherically symmetric may help simplify and improve our understanding of the processes, but are known to be inaccurate. In fact, roughly $80-90\%$ of PNe are not spherical \citep{demarco, soker1997}. \cite{woitke} performed hydrodynamical simulations of dust-driven stellar winds, and concluded that ``highly dynamical and turbulent dust formation [zones]" and flow (e.g. Rayleigh-Taylor) instabilities in stellar atmospheres lead not to symmetric, but to inhomogeneous and asymmetric dust production as seen in the figure below:

\begin{figure}[ht]
    \includegraphics[width=\textwidth]{dust_condensation.png}
\end{figure}

If we accept that Miras are the progenitors to PNe, then then this asymmetry may be explained by the asymmetry of Miras themselves. \cite{ragland} observed, using the IOTA telescope, that for well-resolved stars, ``75\% of AGB stars, 100\% of oxygen-rich Mira stars show asymmetry." However, the causes for asymmetry, and more generally, the astounding complexity and diversity of the morphology of PNe, remain far from absolute. 

One possible explanation lies in binary star systems. The dynamic interactions between two stars of similar or different masses, as well as the physical and temporal nature of the interactions, can all play a role in the formation of PNe. For example, \cite{soker1998} postulates that bipolar nebulae (e.g. M2-9), which are characterized by twin lobes around the CSPN and account for $10-20\%$ of PNe, are formed from ``binary stellar systems in which the secondary diverts a substantial fraction of the mass lost by the asymptotic giant branch (AGB) primary, but the systems avoid the common envelope phase for a large fraction of the interaction time." Binary systems have also been found at the center of the Stingray Nebula, the youngest known PN \citep{bobrowsky}, and at the center of the Red Rectangle, a rectangular protoplanetary nebula \footnote{A protoplanetary nebula (PPN) is an object formed between the transition from AGB to PN. The radiation from the CSPN, which is at this point below the $30,000~{\rm K}$ lower bound required to ionize the gases, is scattered/reflected by the gases, causing the PPN to be visible.} \citep{cohen}.

\section{Summary}\label{sec:summary}


\lipsum[2-3]

Mass loss explains why some intermediate-mass will transition into white dwarfs via the planetary nebulae path rather than exploding as supernovae. Mass loss explains why some intermediate-mass will transition into white dwarfs via the planetary nebulae path rather than exploding as supernovae. 






\nocite{*}
\bibliography{rptNotes}

\end{document}

