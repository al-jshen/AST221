\documentclass[a4paper,10pt, twocolumn]{article}
\usepackage{geometry}
\usepackage{graphicx}
\graphicspath{ {./figures/} }
\usepackage{caption}
\usepackage{enumitem}
\usepackage{multicol}
\usepackage{multirow}
\usepackage{tabularx}
\usepackage{booktabs}
\usepackage{mathtools}
\usepackage{amsmath,amsthm,amssymb,cancel,bm,upgreek}
\usepackage{floatrow}
\usepackage{lastpage}
\geometry{total={210mm,297mm},
left=25mm,right=25mm,%
bindingoffset=0mm, top=20mm,bottom=25mm}
\setlength{\parindent}{0pt}

\newcommand{\linia}{\rule{\linewidth}{0.5pt}}
\newcommand{\fig}[1]{\centerline{\includegraphics[width=0.6\columnwidth]{#1}}}
\AtBeginDocument{%
  \setlength\abovedisplayskip{-3pt}
  \setlength\belowdisplayskip{5pt}}
   

% title configuration
\makeatletter
\def\@maketitle{%
\begin{center}
{\huge \textsc{\@title}\par}
\vspace{1em}
\\
\@author\\
\linia
\vspace{2em}
\end{center}
}
\makeatother

% --------------------------------------------- %
\begin{document}
\title{Formation of Planetary Nebulae\\
    \Large AST221H1 - Fall 2019 | University of Toronto}
\author{Jeff Shen}
\date{\today}
\maketitle

\section{Introduction}
A planetary nebula (abbreviated PN or plural PNe) is an interstellar cloud composed of ionized gas ejected from a low- to intermediate-mass star near the end of its stellar lifetime. 


\subsection{things to write about}
general planetary nebulae stuff

mira variables, oh/ir stars, pulsation theory 

mass loss of agb stars 

agb transition to planetary nebula, protoplanetary nebula

conditions required, temperatures required/reached, timescales of stages

central star 

enrichment 

\subsection{potentially useful sources}
kogan 9.3 (113)

kogan fig 9.48 + caption (142)

kogan 9.3.5, 9.3.6 (124-132)

co 516-519

co example 3.1 (626)

pottasch chapter x (240-270). focus on part f.

\subsection{Readings}

https://www.cfa.harvard.edu/research/oir/planetary-nebulae

https://en.wikipedia.org/wiki/Mira_variable

https://en.wikipedia.org/wiki/Asymptotic_giant_branch

https://en.wikipedia.org/wiki/Protoplanetary_nebula

https://en.wikipedia.org/wiki/Planetary_nebula#Origins

https://web.williams.edu/Astronomy/research/PN/nebulae/nebulaegallery.php


\begin{thebibliography}{9}
    \bibitem{pot}
    Pottasch, S. R.,
    \textit{Planetary Nebulae},
    D. Reidel Publishing Company, 
    1984.

    %\bibitem{kwok}
    %Kwok, Sun,
    %\textit{The Origin and Evolution of Planetary Nebulae},
    %Cambridge University Press,
    %2000.

    \bibitem{co}
    Carroll, B.W., Ostlie, D.A.,
    \textit{An Introduction to Modern Astrophysics},
    Pearson Education Limited, 
    2014.

    \bibitem{kogan}
    Bisnovatyi-Kogan, G.S.,
    \textit{Stellar Physics 2: Stellar Evolution and Stability},
    Springer,
    2011.

\end{thebibliography}

\end{document}
