\documentclass[a4paper,10pt]{article}
\usepackage{geometry}
\usepackage{graphicx}
\graphicspath{ {./figures/} }
\usepackage{caption}
\usepackage{enumitem}
\usepackage{multicol}
\usepackage{multirow}
\usepackage{tabularx}
\usepackage{booktabs}
\usepackage{mathtools}
\usepackage{amsmath,amsthm,amssymb,cancel,bm,upgreek}
\usepackage{floatrow}
\usepackage{lastpage}
\geometry{total={210mm,297mm},
left=25mm,right=25mm,%
bindingoffset=0mm, top=20mm,bottom=25mm}
\setlength{\parindent}{0pt}

\newcommand{\linia}{\rule{\linewidth}{0.5pt}}
\newcommand{\fig}[1]{\centerline{\includegraphics[width=0.6\columnwidth]{#1}}}
\AtBeginDocument{%
  \setlength\abovedisplayskip{-3pt}
  \setlength\belowdisplayskip{5pt}}
   

% title configuration
\makeatletter
\def\@maketitle{%
\begin{center}
\hfill{\textit{Last modified \today}} \\
\vspace{1em}
{\huge \textsc{\@title}\par}
\vspace{1em}
\\
\linia\\
\vspace{1em}
\@author
\vspace{1em}
\end{center}
}
\makeatother

% header, footer configuration
\usepackage{fancyhdr,lastpage}
\pagestyle{fancy}
\lhead{}
\chead{}
\rhead{}
\renewcommand{\headrulewidth}{0pt}
\lfoot{}
\cfoot{}
\rfoot{Page \thepage\ /\ \pageref{LastPage}}

\usepackage[hidelinks]{hyperref}

% --------------------------------------------- %
\begin{document}
\title{AST221: Stars and Planets\\
    \Large University of Toronto | Fall 2019}
\author{Jeff Shen}
\date{\today}
\maketitle
\tableofcontents


% --------------------------------------------- %
% Section 1 %
% --------------------------------------------- %

\newpage
\section{Week 1}

\subsection{Stellar Parallax}

\textbf{Trigonometric parallax}: using a known distance as a baseline, the distance to an object can be determined by observing it from different locations. Measurements of distances to a star can be made on Earth six months apart, when the Sun will have moved a distance of 2 AU (orbital diameter). 

\begin{figure}[h]
    \fig{parallax}
\end{figure}

The parallax angle $p$ is half of the maximum change in position. From this, we can calculate distance as follows:

\begin{align*}
    d = \frac{1}{\tan p\,[{\rm rad}]}~[{\rm AU}] \simeq \frac{1}{p\,[{\rm rad}]}~{\rm AU},
\end{align*}

where for small angles $\tan p \simeq p$ (small-angle approximation). 

Convert this into arcseconds: 

\begin{align*}
    1\,{\rm rad} = 57.3^\circ = 206264.8''
\end{align*}

Defining a new unit called a \textbf{parsec} (parallax-second) as

\begin{align*}
    1\,{\rm pc} = 2.062648 \times 10^5\,{\rm AU} = 3.0856776 \times 10^{16}\,{\rm m},
\end{align*}

we get

\begin{align*}
    d \simeq \frac{1}{p['']}~{\rm pc}.
\end{align*}

In particular, when $p = 1''$, $d = 1\,{\rm pc}$.

\textbf{Light year}: the distance travelled by light through a vacuum in a Julian year: $1\,{\rm ly} = 9.460735\times 10^{15}\,{\rm m} = \frac{1}{3.26}~{\rm pc}$.

\subsection{The Magnitude Scale, Luminosity, and Flux}

\textbf{Apparent magnitude ($m$)}: a logarithmic measure of relative brightness of objects. Brighter objects have a lower $m$ value. Ranges from $m=-26.83$ for the Sun to $m=30$ for the faintest objects in the sky. A 1 mag increase corresponds to a brightness increase of $100^{1/5} \simeq 2.512$, so two stars with a difference of 5 mag differ in brightness by a factor of $100$. 

<<<<<<< HEAD
\textbf{Luminosity ($L$)}: total amount of energy radiated (across all wavelengths) per unit time. Is an intrinsic property, meaning that all observers, regardless of distance from the source, should measure the same luminosity.
=======
\textbf{Luminosity ($L$)}: total amount of energy radiated (across all wavelengths) per unit time. Is an intrinsic property which, for stars, depends on the rate of fusion. $\rm erg\,s^{-1}$. 
>>>>>>> 5b59a2a1c62fc550d1e872e51f06391c8bb4f468

\textbf{Radiant flux ($F$)}: luminosity incident on a unit area oriented perpendicular to the direction of the light. Determines how bright an object is perceived to be. Flux (and radiation in general) follows the \textbf{inverse-square law}, meaning that it is inversely proportional to the square of the distance. Given that stars radiate energy in spheres, and that the area of a sphere is $4\pi r^2$, the flux as measured at some distance $r$ away from a source is:

\begin{align*}
    F = \frac{L}{4\pi r^2}~[{\rm erg\,s^{-1}\,cm^{-2}}].
\end{align*}

\textbf{Absolute magnitude ($M$)}: apparent magnitude of an object, measured at a distance of $10\,{\rm pc}$.

Define the \textbf{flux ratio} as 

\begin{align*}
    \frac{F_2}{F_1} = 100^{(m_1-m_2)/5}~[{\rm dimensionless}].
\end{align*}

Alternatively, taking the logarithm of both sides and rearranging, 

\begin{align*}
    m_1 - m_2 = -2.5\log\left(\frac{F_1}{F_2}\right)~[{\rm dimensionless}].
\end{align*}

Knowing both the apparent and the absolute magnitudes of a star gives us the distance:

\begin{align*}
    d = 10^{(m-M+5)/5}~[{\rm pc}].
\end{align*}

Alternatively, taking the logarithm of both sides and rearranging, we get the \textbf{distance modulus}: 

\begin{align*}
    m-M = 5\log\left(d\right)-5 = 5\log\left({\frac{d}{10\,{\rm pc}}}\right)~[{\rm mag}].
\end{align*}

Constants for the Sun are well known:
\begin{itemize}
    \item $M_\odot = +4.74$
    \item $L_\odot= 3.839\times 10^{33}\,{\rm erg\,s^{-1}}$
\end{itemize}

Thus, the Sun can be used as a reference star in the ratio formulae: 

\begin{align*}
    M &= M_\odot - 2.5\log\left(\frac{L}{L_\odot}\right)~[{\rm mag}],~{\rm and }\\
    m &= M_\odot - 2.5\log\left(\frac{F}{F_{10, \odot}}\right)~[{\rm mag}],
\end{align*}

where $F_{10, \odot}$ is the flux received from the Sun at a distance of $10\,{\rm pc}$.

\subsection{The Copernican Revolution}

\textbf{Heliocentric model}: model of the Solar System proposed by Copernicus. The Sun, rather than the Earth, is placed at the center of the universe, and all other bodies orbit it. Allowed retrograde motion to be better explained. 

\textbf{Inferior planets}: planets orbiting between the Sun and Earth. Mecury, Venus.  

\textbf{Superior planets}: planets with orbits further from the Sun than Earth's. Mars, Jupiter, Saturn, Uranus, Neptune. 

\begin{figure}[h]
    \fig{orbital_configs}
\end{figure}


\textbf{Synodic period ($S$)}: time between successive oppositions or conjunctions 

\textbf{Sidereal period ($P$)}: time to complete one complete orbit as measured relative to background stars

\subsection{Equations}
\vspace{0.5cm}

\begin{tabularx}{\textwidth}{ l X r }
    Parallax && $d = \frac{1}{\tan p\,[{\rm rad}]}~[{\rm AU}] \simeq \frac{1}{p\,[{\rm rad}]}~{\rm AU} \simeq \frac{206264}{p['']}~[{\rm AU}] = \frac{1}{p['']}~[{\rm pc}]$ \\ 
    \addlinespace[0.5cm]
    Flux and luminosity && $F = \frac{L}{4\pi r^2}~[{\rm erg\,s^{-1}\,cm^{-2}}]$ \\
    \addlinespace[0.5cm]
    Flux ratio && $\frac{F_2}{F_1} = 100^{(m_1-m_2)/5}~[{\rm dimensionless}]$ \\
    \addlinespace[0.5cm]
               && $m_1 - m_2 = -2.5\log\left(\frac{F_1}{F_2}\right)~[{\rm dimensionless}]$ \\
    \addlinespace[0.5cm]
               && $M = M_\odot - 2.5\log\left(\frac{L}{L_\odot}\right)~[{\rm mag}]$ \\
    \addlinespace[0.5cm]
               && $m = M_\odot - 2.5\log\left(\frac{F}{F_{10, \odot}}\right)~[{\rm mag}]$ \\
    \addlinespace[0.5cm]
    Distance && $d = 10^{(m-M+5)/5}~[{\rm pc}]$ \\
    \addlinespace[0.5cm]
    Distance modulus && $m-M = 5\log\left(d\right)-5 = 5\log\left({\frac{d}{10\,{\rm pc}}}\right)~[{\rm mag}]$ \\ 
    \addlinespace[0.5cm]
                     && $m-M = 5\log\left(d\right)-5 = 5\log\left({\frac{d}{10\,{\rm pc}}}\right)~[{\rm mag}]$
\end{tabularx}



% --------------------------------------------- %
% Week 2 %
% --------------------------------------------- %

\newpage
\section{Week 2}

\subsection{Orbits}

Orbital shapes are always \textbf{conic sections}. The shape determines the type of orbit, and the shape in turn is determined by the \textbf{eccentricity $e$}—which is defined as the distance between the foci divided by the major axis—of the orbit. 

For $e<1$, we have \textbf{elliptical orbits}:

\fig{ellipse}

An ellipse is defined by the set of points satisfying the equation

\begin{align*}
    r + r' = 2a
\end{align*}

where $r, r'$ are the distances from the ellipse to the focal points $F$ (the \textbf{principal focus}) and $F'$ respectively, and $a$ is the \textbf{semi-major axis}. The distance between each focus and the center of the ellipse is given by $ae$. The \textbf{semi-minor axis} is given by the distance $b$. 

Consider a point at the end of a semi-minor axis, where $r=r'$. By definition of an ellipse, we then know that 

\begin{align*}
    r+r'=r+r=2r=2a
    \implies r=a.
\end{align*}

By the Pythagorean Theorem, we know that 

\begin{align*}
    (ae)^2 + b^2 = r^2.
\end{align*}

Putting these two equations together, we find that 

\begin{align*}
    b^2 = a^2(1-e^2).
\end{align*}

If we consider the triangle in the above diagram, we can derive an expression for $r'$ in terms of $a, e, r$, and $\theta$:

\begin{align*}
    (r')^2 &= x^2 + y^2 \\
           &= (2ae + r\cos\theta)^2 + (r\sin\theta)^2 \\
           &= 4a^2e^2 + 4aer\cos\theta + r^2\cos^2\theta + r^2\sin^2\theta \\
           &= 4a^2e^2 + 4aer\cos\theta + r^2 \\
           &= 4ae(ae + r\cos\theta) + r^2.
\end{align*}

Using the definition of an ellipse, $r+r'=2a$, we have 

\begin{align*}
    r = \frac{a(1-e^2)}{1+e\cos\theta}
\end{align*}

\textbf{Perihelion} is the point on the ellipse which is closest to the principal focus, and is located at $\theta=0$. \textbf{Aphelion} is the point on the opposite end of the ellipse (furthest point from the principal focus), and is located at $\theta=180^\circ$. Thus, distances for perihelion and aphelion are given by 

\begin{alignat*}{2}
    &\text{Perihelion} &&\text{Aphelion} \\[0.2cm]
    r_p &= \frac{a(1-e^2)}{1+e\cos 0} &\qquad\qquad\qquad r_a &= \frac{a(1-e^2)}{1+e\cos 180^\circ} \\
        &= \frac{a(1-e^2)}{1+e} &&= \frac{a(1-e^2)}{1-e} \\
        &= a(1-e). &&= a(1+e).
\end{alignat*}

A curve with eccentricity $e=1$ is called a \textbf{parabola}, and is described by the equation 

\begin{align*}
    r = \frac{2p}{1+\cos\theta}.
\end{align*}

where $p$ is the distance of closest appoach to the parabola's single focus at $\theta=0$. Parabolic trajectories are minimum-energy escape trajectories. 

A curve with eccentricity $e>1$ is called a \textbf{hyperbola}, and is described by the equation

\begin{align*}
    r = \frac{a(e^2-1)}{1+e\cos\theta}.
\end{align*}

\subsection{Newtonian Mechanics}

\textbf{Newton's Laws of Motion}
\begin{enumerate}
    \item An object at rest will remain at rest and an object in motion will remain in motion in a straight line at a constant speed unless acted upon by an external force.
    \item The net force (the sum of all forces) acting on an object is proportional to the object’s mass and its resultant acceleration. If an object is experiencing $n$ forces, then the net force is given by

        \begin{align*}
            \vec{F}_{\rm net} = \sum_{i=1}^n \vec{F}_i = m\vec{a}~[{\rm N}].
        \end{align*}

    \item For every action there is an equal and opposite reaction. If object 1 exerts a force $\vec{F}_{12}$ on some other object, then object 2 must exert a force $\vec{F}_{21}$ on object 1 of equal magnitude and opposite direction. Mathematically, 

        \begin{align*}
            \vec{F}_{12} = -\vec{F}_{21}~[{\rm N}].
        \end{align*}
\end{enumerate}

\textbf{Law of Universal Gravitation}: this is the force responsible for holding planets in orbit. More generally, it applies to any two objects with mass: 

\begin{align*}
    F = G\frac{Mm}{r^2}~[{\rm N}]
\end{align*}

where $G=6.67\times 10^{-11}\,{\rm N\,m^2\,kg^{-2}}$ is the \textbf{universal gravitational constant}.

{\Huge derivations of gravitational potential energy, work=change in kinetic energy, escape velocity IF HAVE TIME}

\subsection{Kepler's Laws of Planetary Motion}

{\Huge do derivations!}

\subsubsection{N-Body Orbits}
\subsubsection{First Law}

\begin{align*}
    |r| = \frac{|L|^2/\mu^2}{GM(1+e\cos\theta)}~[{\rm m}]
\end{align*}

\begin{align*}
    |L| = \mu\sqrt{GMa(1-e^2)}~[{\rm kg\,m^2\,s^{-1}}]
\end{align*}

\subsubsection{Second Law}

\begin{align*}
    \frac{{\rm d}A}{{\rm d}t} = \frac{1}{2}\frac{L}{\mu}~[\rm m^2\,s^{-1}]
\end{align*}

\subsubsection{Third Law}

\begin{align*}
    P^2 = \frac{4\pi^2a^3}{G(m_1+m_2)}
\end{align*}

\subsection{Equations}

\begin{tabularx}{\textwidth}{ l X r }

    Orbital equations && $r + r' = 2a$, for $0\leq e<1$ \\[0.3cm]
                      && $r = \frac{2p}{1+\cos\theta}$, for $e=1$ \\[0.3cm]
                      && $r = \frac{a(e^2-1)}{1+e\cos\theta}$, for $e>1$ \\[0.3cm]

    Orbital position && $r = \frac{a(1-e^2)}{1+e\cos\theta}$, for $0\leq e<1$ \\[0.3cm]

    Newton's Laws of Motion && Inertia \\[0.3cm]
                            &&$\vec{F}_{\rm net} = \sum_{i=1}^n \vec{F}_i = m\vec{a}~[{\rm N}]$ \\[0.3cm]
                            && $\vec{F}_{12} = -\vec{F}_{21}~[{\rm N}]$ \\[0.3cm]
    
    Law of Universal Gravitation && $F = G\frac{Mm}{r^2}~[\rm N]$ \\[0.3cm]

    Kepler's Laws of Planetary Motion && $|r| = \frac{|L|^2/\mu^2}{GM(1+e\cos\theta)}~[{\rm m}]$ \\[0.3cm]
                                      && $\frac{{\rm d}A}{{\rm d}t} = \frac{1}{2}\frac{L}{\mu}~[\rm m^2\,s^{-1}]$ \\[0.3cm]
                                      && $P[\rm years]^2 = \frac{4\pi^2}{G(m_1+m_2)}a[\rm AU]^3$ \\[0.3cm]
\end{tabularx}




% --------------------------------------------- %
% Week 3 %
% --------------------------------------------- %
\newpage
\section{Week 3}

\subsection{Tides and Moons}

\subsection{Equations}

% --------------------------------------------- %
% Week 4 %
% --------------------------------------------- %

\newpage
\section{Week 3}

\subsection{Hydrostatic Equilibrium}

\subsection{The Virial Theorem}

\subsection{Equations}


% --------------------------------------------- %
% Week 5 %
% --------------------------------------------- %

\newpage
\section{Week 5}

\subsection{Nuclear Fusion}

\subsection{Blackbody Radiation}

\subsection{Spectral Lines}
quantization
doppler

\subsection{Light}

\subsection{Photon Diffusion}
mfp

\subsection{Equations}


% --------------------------------------------- %
% Week 6 %
% --------------------------------------------- %

\newpage
\section{Week 6}

\subsection{Stellar Evolution: Pre-MS}

\subsection{Stellar Evolution: MS}
mass, size, brightness relations

\subsection{Timescales}

\subsection{Equations}

 
% --------------------------------------------- %
% Week 7 %
% --------------------------------------------- %

\newpage
\section{Week 7}

\subsection{White Dwarfs}

\subsection{Electron Degeneracy}

\subsection{Equations}

% --------------------------------------------- %
% Week 8 %
% --------------------------------------------- %
\newpage
\section{Week 8}

\subsection{Stellar Evolution: Post-MS}

\subsection{Neutron Stars}

\subsection{Black Holes}

\subsection{Equations}


\end{document}
