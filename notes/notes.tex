\documentclass[a4paper,10pt]{article}
\usepackage{geometry}
\usepackage{graphicx}
\graphicspath{ {./figures/} }
\usepackage{caption}
\usepackage{enumitem}
\usepackage{multicol}
\usepackage{multirow}
\usepackage{booktabs}
\usepackage{mathtools}
\usepackage{amsmath,amsthm,amssymb,cancel,bm,upgreek}
\usepackage{floatrow}
\usepackage{lastpage}
\geometry{total={210mm,297mm},
left=25mm,right=25mm,%
bindingoffset=0mm, top=20mm,bottom=20mm}
\setlength{\parindent}{0pt}

\newcommand{\linia}{\rule{\linewidth}{0.5pt}}
\newcommand{\fig}[1]{\centerline{\includegraphics{#1}}}
\AtBeginDocument{%
  \setlength\abovedisplayskip{-3pt}
  \setlength\belowdisplayskip{5pt}}
   

% title configuration
\makeatletter
\def\@maketitle{%
\begin{center}
\hfill{\textit{Last modified \today}} \\
\vspace{1em}
{\huge \textsc{\@title}\par}
\vspace{1em}
\\
\linia\\
\vspace{1em}
\@author
\vspace{1em}
\end{center}
}
\makeatother

% header, footer configuration
\usepackage{fancyhdr,lastpage}
\pagestyle{fancy}
\lhead{}
\chead{}
\rhead{}
\renewcommand{\headrulewidth}{0pt}
\lfoot{AST221: Stars and Planets}
\cfoot{}
\rfoot{Page \thepage\ /\ \pageref{LastPage}}

\usepackage[hidelinks]{hyperref}

% --------------------------------------------- %
\begin{document}
\title{AST221: Stars and Planets\\
    \Large University of Toronto | Fall 2019}
\author{Jeff Shen}
\date{\today}
\maketitle
\tableofcontents


% --------------------------------------------- %
% Section 1 %
% --------------------------------------------- %

\newpage
\section{Week 1}

\subsection{Stellar Parallax}

Trigonometric parallax: using a known distance as a baseline, the distance to an object can be determined by observing it from different locations. Measurements of distances to a star can be made on Earth six months apart, when the Sun will have moved a distance of 2 AU (orbital diameter). 

\begin{figure}[h]
    \fig{parallax}
\end{figure}

The parallax angle $p$ is half of the maximum change in position. From this, we can calculate distance as follows:

\begin{align*}
    d = \frac{1 {\rm AU}}{\tan p\,[{\rm rad}]} \simeq \frac{1}{p\,[{\rm rad}]} ~ {\rm AU},
\end{align*}

where for small angles $\tan p \simeq p$ (small-angle approximation). 

Convert this into arcseconds: 
\begin{align*}
    1\,{\rm rad} = 57.3^\circ = 206264.8''
\end{align*}

Defining a new unit called a \textbf{parsec} (parallax-second) as

\begin{align*}
    1\,{\rm pc} = 2.062648 \times 10^5\,{\rm AU} = 3.0856776 \times 10^{16}\,{\rm m},
\end{align*}

we get

\begin{align*}
    d \simeq \frac{1}{p['']} ~ {\rm pc}.
\end{align*}

In particular, when $p = 1''$, $d = 1\,{\rm pc}$.

\textbf{Light year}: the distance travelled by light through a vacuum in a Julian year: $1\,{\rm ly} = 9.460735\times 10^{15}\,{\rm m} = \frac{1}{3.26}~{\rm pc}$.

\subsection{The Magnitude Scale, Luminosity, and Flux}

\textbf{Apparent magnitude ($m$)}: a logarithmic measure of relative brightness of objects. Brighter objects have a lower $m$ value. Ranges from $m=-26.83$ for the Sun to $m=30$ for the faintest objects in the sky. A 1 mag increase corresponds to a brightness increase of $100^{1/5} \simeq 2.512$. Dimensionless. 

\textbf{Luminosity ($L$])}: total amount of energy radiated (across all wavelengths) per unit time. Is an intrinsic property which, for stars, depends on the rate of fusion. $\rm erg\,s^{-1}$. 

\textbf{Radiant flux ($F$)}: luminosity incident on a unit area oriented perpendicular to the direction of the light. Determines how bright an object is perceived to be. Flux (and radiation in general) follows the \textbf{inverse-square law}, meaning that it is inversely proportional to the square of the distance. Given that stars radiate energy in spheres, and that the area of a sphere is $4\pi r^2$, the flux as measured at some distance $r$ away from a source is:

\begin{align*}
    F = \frac{L}{4\pi r^2}~[{\rm erg\,s^{-1}\,cm^{-2}}].
\end{align*}

\textbf{Absolute magnitude ($M$)}: apparent magnitude of an object, measured at a distance of $10\,{\rm pc}$.

Define the \textbf{flux ratio} as 

\begin{align*}
    \frac{F_2}{F_1} = 100^{(m_1-m_2)/5}~[{\rm dimensionless}].
\end{align*}

Alternatively, taking the logarithm of both sides and rearranging, 

\begin{align*}
    m_1 - m_2 = -2.5\log\left(\frac{F_1}{F_2}\right)~[{\rm dimensionless}].
\end{align*}

Knowing both the apparent and the absolute magnitudes of a star gives us the distance (\textbf{distance modulus}):

\begin{align*}
    d = 10^{(m-M+5)/5}~[{\rm pc}].
\end{align*}

Alternatively, taking the logarithm of both sides and rearranging, 

\begin{align*}
    m-M = 5\log\left(d\right)-5 = 5\log\left({\frac{d}{10\,{\rm pc}}}\right)~[{\rm mag}].
\end{align*}





\subsection{The Copernican Revolution}

\subsection{Definitions and Equations}

% --------------------------------------------- %
% Week 2 %
% --------------------------------------------- %

\newpage
\section{Week 2}

\subsection{Orbital Mechanics}

\subsection{Newtonian Mechanics}

\subsection{Kepler's Laws of Planetary Motion}
derivations
\subsubsection{N-Body Orbits}
\subsubsection{First Law}
\subsubsection{Second Law}
\subsubsection{Third Law}


% --------------------------------------------- %
% Week 3 %
% --------------------------------------------- %
\newpage
\section{Week 3}

\subsection{Tides and Moons}

\subsection{Definitions and Equations}

% --------------------------------------------- %
% Week 4 %
% --------------------------------------------- %

\newpage
\section{Week 3}

\subsection{Hydrostatic Equilibrium}

\subsection{The Virial Theorem}

\subsection{Definitions and Equations}


% --------------------------------------------- %
% Week 5 %
% --------------------------------------------- %

\newpage
\section{Week 5}

\subsection{Nuclear Fusion}

\subsection{Blackbody Radiation}

\subsection{Spectral Lines}
quantization
doppler

\subsection{Light}

\subsection{Photon Diffusion}
mfp

\subsection{Definitions and Equations}


% --------------------------------------------- %
% Week 6 %
% --------------------------------------------- %

\newpage
\section{Week 6}

\subsection{Stellar Evolution: Pre-MS}

\subsection{Stellar Evolution: MS}
mass, size, brightness relations

\subsection{Timescales}

\subsection{Definitions and Equations}

 
% --------------------------------------------- %
% Week 7 %
% --------------------------------------------- %

\newpage
\section{Week 7}

\subsection{White Dwarfs}

\subsection{Electron Degeneracy}

\subsection{Definitions and Equations}

% --------------------------------------------- %
% Week 8 %
% --------------------------------------------- %
\newpage
\section{Week 8}

\subsection{Stellar Evolution: Post-MS}

\subsection{Neutron Stars}

\subsection{Black Holes}

\subsection{Definitions and Equations}


\end{document}
