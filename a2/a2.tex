\documentclass[11pt,letterpaper]{article}
\usepackage{fullpage}
\usepackage[top=2cm, bottom=4.5cm, left=2.5cm, right=2.5cm]{geometry}
\usepackage{amsmath,amsthm,amsfonts,amssymb,amscd}
\usepackage{lastpage}
\usepackage{enumerate}
\usepackage{enumitem}
\usepackage{fancyhdr}
\usepackage{mathrsfs}
\usepackage{mathabx}
\usepackage{siunitx}
\usepackage{graphicx}
\usepackage{listings}
\usepackage{hyperref}
\usepackage{booktabs}
\usepackage{caption,cleveref,colortbl,csquotes,datatool,helvet,mathpazo,multirow,listings,pgfplots,xcolor}

\DeclareSIUnit \erg{erg}
\DeclareSIUnit \au{au}
\DeclareSIUnit \pc{pc}
\DeclareSIUnit \ly{ly}
\DeclareSIUnit \years{years}
\DeclareSIUnit \K{K}

\hypersetup{%
  colorlinks=true,
  linkcolor=blue,
  linkbordercolor={0 0 1}
}

\setlength{\parindent}{0.0in}
\setlength{\parskip}{0.05in}

% edit these
\newcommand\course{AST221H}
\newcommand\Title{Assignment 2}
\newcommand\Name{Jeff Shen} 
\newcommand\Id{1004911526} 
\newcommand\Date{Oct. 4, 2019}

\pagestyle{fancyplain}
\headheight 35pt
\lhead{\Name}
\lhead{\Name\\\Id}
\chead{\textbf{\Large \Title}}
\rhead{\course \\ \Date}
\lfoot{}
\cfoot{}
\rfoot{\small\thepage}
\headsep 1.5em

\begin{document}

% problem 1
\section*{Problem 1: Tides}

\begin{enumerate}[label=(\alph*)]
    \item
    We want to find the relative tidal force on Earth due to the Sun and Moon. Tidal force is given by $\Delta F = -\frac{2GMm}{r^3}dr.$ Then we have $$\frac{\Delta F_{\Moon\Earth}}{\Delta F_{\Sun\Earth}} = \frac{-\frac{2GM_\Moon M_\Earth}{r_{\Moon\Earth}^3}dr}{-\frac{2GM_\Sun M_\Earth}{r_{\Sun\Earth}^3}dr} = \frac{\frac{M_\Moon}{r_{\Moon\Earth}^3}}{\frac{M_\Sun}{r_{\Sun\Earth}^3}} = \frac{M_\Moon r_{\Sun\Earth}^3}{M_\Sun r_{\Moon\Earth}^3},$$
    where $\Delta F_{\Moon\Earth}$ and $\Delta F_{\Sun\Earth}$ are the tidal forces on Earth due to the Moon and the Sun respectively, $r_{\Sun\Earth}$ and $r_{\Moon\Earth}$ are the distances between the Earth and the Sun and Moon respectively, and $M_\Earth$, $M_\Sun$, and $M_\Moon$ are the masses of the Earth, Sun, and Moon respectively. So we have shown that the relative tidal force depends only on $M_\Moon$, $M_\Sun$, $r_{\Sun\Earth}$, and $r_{\Moon\Earth}$. 
    
    In the appendix, the following values are given:
    \begin{itemize}
        \item $M_\Moon = 7.349\times 10^{22}\;\si{kg}$
        \item $M_\Sun = 1.99\times 10^{30}\;\si{kg}$
        \item $r_{\Sun\Earth} = 1\;\si{AU} = 1.496\times 10^{11}\;\si{m}$
        \item $r_{\Moon\Earth} = 3.844\times 10^8\;\si{m}$
    \end{itemize}
  
    Plugging these values into the equation above, we find that the tidal force due to the Moon is approximately twice as strong as the tidal force due to the Sun: $$\frac{\Delta F_{\Moon\Earth}}{\Delta F_{\Sun\Earth}} = \frac{M_\Moon r_{\Sun\Earth}^3}{M_\Sun r_{\Moon\Earth}^3} = \frac{(7.349\times 10^{22}\;\si{kg})(1.496\times 10^{11}\;\si{m})^3}{(1.99\times 10^{30}\;\si{kg})(3.844\times 10^8\;\si{m})^3} \approx 2.177$$
    
    \item
    The tidal force per unit mass at the surface of Io facing Jupiter is given by $$\frac{\Delta F}{m} = -\frac{2GM_JR_I}{(r_{J,I} - R_I)^3}.$$ where $M_J$ is the masses of Jupiter, $R_I$ is the radius of Io, and $r_{J, I}$ is the distance from Jupiter to the surface of Io (given by the difference between the distance from Jupiter to Io and the radius of Io).
    
    The gravitational force per unit mass due to Io at the same location is given by $$\frac{F_G}{m} = -\frac{GM_I}{R_I^2},$$ where $M_I$ is the mass of Io and $R_I$ is the radius of Io. 
    
    Expressing the two forces as a ratio, we find that $$\frac{\Delta F}{F_G} = \frac{-\frac{2GM_JR_I}{(r_{J,I} - R_I)^3}}{-\frac{GM_I}{R_I^2}} = \frac{-2GM_JR_I^3}{-GM_I(r_{J,I} - R_I)^3} = \frac{2M_JR_I^3}{M_I(r_{J,I} - R_I)^3}.$$ Substituting in the appropriate values into the equation, we find that $$\frac{\Delta F}{F_G} = \frac{2\times 1.898\times 10^{27}\;\si{kg}\times (1.8216\times 10^6\;\si{m})^3}{8.932\times 10^{22}\;\si{kg}\times (4.216\times 10^8\;\si{m} - 1.8216\times 10^6\;\si{m})^3} = 0.00347 = 3.47\times 10^{-3}.$$ This suggests that $F_{tide}/F_{grav}$ for tides Io due to Jupiter is much stronger than for tides on Earth due to the Moon. 

    \item
    We take the tidal force formula and remove the smaller mass to get the tidal acceleration: $\Delta a = -\frac{2GM}{r^3}dr$. For $r$, we use the distance from the center of mass of the Moon to your center of mass (which is approximately at the surface of the Earth). So then the distance is $$r \approx r_{\Moon\Earth} - R_\Earth = 3.844\times 10^8\;\si{m} - 6.378\times 10^6\;\si{m} = 3.78\times 10^8\;\si{m},$$ where $R_\Earth$ is the radius of the Earth. Plugging this distance into the tidal acceleration equation along with the mass of the Moon as $M$ and the distance between your toes and your head as $dr$, we can find the tidal acceleration due to the Moon: $$\Delta a = -\frac{2\times 6.67\times 10^{-11}\;\si{\newton\square\meter\per\square\kilogram}\times 7.349\times 10^{22}\;\si{kg} \times 1.7\;\si{m}}{(3.78\times 10^8\;\si{m})^3} = -3.09\times 10^{-13}\;\si{\meter\per\square\second}.$$
    
    To calculate the tidal acceleration due to the Earth, we use the same formula, but substitute in the appropriate values (radius of the Earth for $r$, and mass of the Earth for $M$): $$\Delta a = -\frac{2\times 6.67\times 10^{-11}\;\si{\newton\square\meter\per\square\kilogram}\times 5.97\times 10^{24}\;\si{kg} \times 1.7\;\si{m}}{(6.378\times 10^6\;\si{m})^3} = -5.22\times 10^{-6}\;\si{\meter\per\square\second}.$$ So the tidal acceleration due to the Moon is approximately $10^{-7}$ times weaker than the tidal acceleration due to the Earth. 
    
    The overall gravitational acceleration due to Earth's gravity is $-9.8\;\si{\meter\per\square\second}.$ So the tidal acceleration due to the Moon is (very) approximately $10^{-14}$ times weaker. 
    
    \item
    A 100 million solar mass black hole would have mass $10^8\times 1.99\times 10^{30}\;\si{\kg} = 1.99\times 10^{38}\;\si{\kg}.$
    The Schwarzchild radius of a black hole is given by the formula $$R_S = \frac{2GM}{c^2},$$ where $c$ is the speed of light, so a 100 million solar mass black hole would have Schwarzchild radius $$R_S = \frac{2\times 6.67\times 10^{-11}\;\si{\newton\square\meter\per\square\kg}\times 1.99\times 10^{38}\;\si{\kg}}{(3.00\times 10^8\;\si{\meter\per\second})^2} = 2.95\times 10^{11}\;\si{\meter}.$$ Using $2\;\si{\meter}$ as the height of the astronaut, they would experience a tidal acceleration of $$\Delta a = -\frac{2GM}{r^3}dr = -\frac{-2\times 6.67\times 10^{-11}\;\si{\newton\square\meter\per\square\kg}\times 1.99\times 10^{38}\;\si{\kg}\times 2\;\si{\meter}}{(2.95\times 10^{11}\;\si{\meter})^3} = -2.06\times 10^{-6}\;\si{\meter\per\square\second}.$$ It seems reasonable that an astronaut could survive this acceleration given that the acceleration due to Earth's gravity is $-9.8\;\si{\meter\per\square\second}.$ 
    
    For a 10 solar mass black hole, we can perform the same calculations with the lower mass: $$R_S = \frac{2\times 6.67\times 10^{-11}\;\si{\newton\square\meter\per\square\kg}\times 1.99\times 10^{31}\;\si{\kg}}{(3.00\times 10^8\;\si{\meter\per\second})^2} = 2.95\times 10^{14}\;\si{\meter}.$$
    
    $$\Delta a = -\frac{2GM}{r^3}dr = -\frac{-2\times 6.67\times 10^{-11}\;\si{\newton\square\meter\per\square\kg}\times 1.99\times 10^{31}\;\si{\kg}\times 2\;\si{\meter}}{(2.95\times 10^4\;\si{\meter})^3} = -2.06\times 10^8\;\si{\meter\per\square\second}.$$ This is approximately $10^7$ times the acceleration due to Earth's gravity, so the astronaut would probably be spaghettified. 
    

\end{enumerate}

% problem 2
\section*{Problem 2: Virial theorem}

\begin{enumerate}[label=(\alph*)]

    \item
    For a relativistic gas, we know that $$ P = \frac{E_{th}}{3V} \implies PV=\frac{E_{th}}{3}.$$ We also have the ideal gas law, which says that $PV=NkT,$ where $N$ is the number of gas particles. Then substituting this in to the previous equation, we find that $$\frac{E_{th}}{3} = PV = NkT \implies E_{th} = 3NkT.$$ We can find the total thermal energy $E_{th}^{tot}$ by integrating over the volume of a star: $$E_{th}^{tot} = \int3NkTdV = 3\int NkTdV.$$ This integral is equivalent to integrating the pressure on some shell (with spherical area given by $4\pi r^2$), while varying the radius of that shell from 0 to the outer radius, $R$, of the star: $$3\int3NkTdV = 3\int_{0}^{R}P\cdot 4\pi r^2 dr.$$ But then the integral on the right is just the mean pressure multiplied by the total volume of the star. So then $$3\int_{0}^{R}P\cdot 4\pi r^2 dr = 3\overline{P}V.$$ Finally, substituting in  $\overline{P} = -\frac{E_{grav}}{3V}$, we see that $$3\overline{P}V = 3(-\frac{E_{grav}}{3V})V = -E_{grav}.$$ So $$E_{th}^{tot} = \int3NkTdV = 3\int_{0}^{R}P\cdot 4\pi r^2 dr = 3\overline{P}V = -E_{grav}.$$
    
    
    \item
    By the Virial Theorem, we know know that the pressure and gravitational energy of a star are related by the following expression: $$P = -\frac{E_{grav}}{3V}.$$ We can approximate $E_{grav}$ for a star with mass $M$ and radius $R$ by $E_{grav} \approx -\frac{GM^2}{R}.$ Then we substitute this into the previous expression to get: $$P = -\frac{-\frac{GM^2}{R}}{3V} = \frac{GM^2}{3VR}.$$ We can take the cube of both sides to simplify calculations: $$P^3 = \frac{G^3M^6}{3^3V^3R^3} = \frac{G^3M^2M^4}{3^3V^3R^3}.$$ But note that we can relate the mass, density, and volume as follows: $$M = \rho V.$$ We can also express volume of a (spherical) star in terms of its radius: $$V = \frac{4\pi}{3}R^3.$$ We substitute these two relations into the pressure equation while cancelling terms along the way to get: $$P^3 = \frac{G^3M^2(\rho V)^4}{3^3V^3R^3} = \frac{G^3M^2\rho^4V}{3^3R^3} = \frac{G^3M^2\rho^4(\frac{4\pi}{3}R^3)}{3^3R^3} = \frac{4\pi G^3M^2\rho^4}{3^4}.$$ Then we take the cube root of both sides to obtain the relation that we want: $$\sqrt[3]{P^3} = P = \sqrt[3]{\frac{4\pi G^3M^2\rho^4}{3^4}} = (\frac{4\pi}{3^4})^{1/3}GM^{2/3}\rho^{4/3}.$$
    
    
    \item
    If radiation pressure equals kinetic pressure, then we have that the total pressure is
    $$P = P_{kin} + P_{rad} = \frac{\rho kT}{\overline{m}} + \frac{1}{3}aT^4 = 2\frac{\rho kT}{\overline{m}}.$$
    We can also use this fact to express T in terms of other variables: $$\frac{\rho kT}{\overline{m}} = \frac{1}{3}aT^4 = \frac{1}{3}aT^3T \implies \frac{\rho k}{\overline{m}} = \frac{1}{3}aT^3 \implies T^3 = (\frac{3}{a})(\frac{\rho k}{\overline{m}}) \implies T = (\frac{3}{a})^{1/3}(\frac{\rho k}{\overline{m}})^{1/3}.$$ Substituting this back into the previous equation, we find that $$P = 2\frac{\rho k}{\overline{m}}T = 2\frac{\rho}{\overline{m}}(\frac{3}{a})^{1/3}(\frac{\rho k}{\overline{m}})^{1/3} = 2(\frac{3}{a})^{1/3}(\frac{\rho k}{\overline{m}})^{4/3},$$ which is what we wanted to show.
    
    
    \item
    We equate the two expressions for pressure: 
    $$P = (\frac{4\pi}{3^4})^{1/3}GM^{2/3}\rho^{4/3} = 2(\frac{3}{a})^{1/3}(\frac{k\rho}{\overline{m}})^{4/3}$$ and then cube both sides to get $$\frac{4\pi}{3^4}G^3M^2\rho^4 = 8\frac{3}{a}\frac{k^4\rho^4}{\overline{m}^4}.$$ Then we isolate the $M^2$ term, and consolidate and cancel terms: $$M^2 = (\frac{8 \cdot 3 k^4 \rho^4}{a\overline{m}^4})(\frac{3^4}{4\pi G^3\rho^4}) = \frac{2\cdot 3^5 k^4}{\pi a \overline{m}^4 G^3}.$$ Taking the square root gives us $$M = \sqrt{\frac{2\cdot 3^5 k^4}{\pi a \overline{m}^4 G^3}}.$$ We can evaluate this using the following values:
    \begin{itemize}
        \item Boltzmann constant: $k = 1.381\times 10^{-23}\;\si{J\;K^{-1}}$
        \item radiation constant: $a = \frac{4\sigma}{c} = 7.566\times 10^{-16}\;\si{J\;m^{-3}\;K^{-4}}$, where $\sigma$ is the Stefan-Boltzmann constant and $c$ is the speed of light
        \item gravitational constant: $G = 6.674 \times 10^{-11} \;\si{N\;m^2\;kg^{-2}}$
        \item mean particle mass (mass of ionized hydrogen): $\overline{m} = \frac{1}{2}m_H = \frac{1}{2} \cdot 1.674\times 10^{-27}\;\si{kg} = 8.369\times 10^{-22}\;\si{kg}$
    \end{itemize}
    to find that the maximum mass of the star is $2.257 \times 10^{32}\;\si{kg}.$ This is approximately 113.5 $M_\odot$.
    
\end{enumerate}





\section*{Problem 3: The Sun and neutrinos}
\begin{enumerate}[label=(\alph*)]
    \item 
    Energy from one cycle of p-p chain: $26.2\times 10^{26} \;\si{MeV} = 4.20\times 10^{-12}\;\si{J/cycle}$
    
    Luminosity of the Sun: $3.828 \times 10^{26} \;\si{J/s}$
    
    Then divide to find cycles per second: $$\frac{3.828 \times 10^{26} \;\si{J/s}}{4.20\times 10^{-12}\;\si{J/cycle}} = 9.11\times 10^{37}\;\si{cycles/s}$$
    
    \item
    Use $E=mc^2$ to find the energy released if 10\% of mass of each tower is converted to energy: $$E=(7.3\times 10^6\;\si{kg}) (3.0\times 10^8\;\si{m/s})^2 = 6.57\times 10^{23}\;\si{J/tower}$$
    
    Then again, we use the luminosity of the sun to find the number of towers: $$\frac{3.828\times 10^{26}\;\si{J/s}}{6.57\times 10^{23}\;\si{J/tower}} \approx 583\;\si{towers/s}$$
    
    \item
    We first calculate the number of neutrinos that are produced each second: $$9.11\times 10^{37}\;\si{cycles/s}\cdot 2\;\si{neutrinos/cycle} = 1.822\times 10^{38}\;\si{neutrinos/s}.$$ If we assume that neutrinos fly out in all directions with equal probability, then we divide by the area of a sphere: $4\pi r^2.$ We take $r$ to be the distance from the center of the Sun to our position at the surface of the Earth, but since the radius of the Earth is negligible compared to the distance from the Sun to the Earth, we can use 1 au as $r$. Thus, the particle flux of neutrinos on Earth is given by $$\frac{1.822\times 10^{38}\;\si{neutrinos/s}}{4\pi(1.496\times 10^{13}\;\si{cm})^2} = 6.44\times 10^{10}\;\si{neutrinos/s/cm^2}.$$ If we assume our that the brain is spherical, and approximate its volume to be $1300\;\si{cm^3}$ (various Internet sources), then we can find the radius of the brain: $$V = \frac{4\pi}{3}r^3 \implies r = \sqrt[3]{\frac{3V}{4\pi}} = \sqrt[3]{\frac{3\cdot 1300\;\si{cm^3}}{4\pi}} = 6.77\;\si{cm}.$$ Using this, we assume that half of the brain faces towards the Sun, so half of its surface area is being hit by neutrinos: $$\frac{1}{2} A_S = \frac{1}{2} \cdot 4\pi (6.77\;\si{cm})^2 = 288\;\si{cm^2}$$ 
    
    Thus, the number of neutrinos hitting the brain each second is (very) approximately $$6.44\times 10^{12}\;\si{neutrinos/s/cm^2} \cdot 288\;\si{cm^2} = 1.85\times 10^{15}\;\si{neutrinos/s}.$$
    
\end{enumerate}

\end{document}
